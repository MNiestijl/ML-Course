\documentclass [a4paper] {report}
\usepackage{amsmath,amssymb,amsthm, bbm, graphicx,listings,braket,subfig,titlesec,cleveref,lipsum,mcode,xcolor-patch, textcomp,float,booktabs,siunitx, listings}
\usepackage[authoryear]{natbib}
\usepackage[section]{placeins}
\usepackage[margin=2.2cm]{geometry}
\titleformat{\chapter}{\normalfont\huge}{\thechapter.}{20pt}{\huge \bf}

\DeclareMathOperator*{\argmin}{arg\,min}
\DeclareMathOperator*{\argmax}{arg\,max}
\newcommand{\norm}[1]{\left\lVert #1 \right\rVert}

\begin{document}
	
	\begin{titlepage}
		\begin{center}
			
			\textsc{\LARGE IN4320 Machine Learning}\\[1.25cm]
			
			\rule{\linewidth}{0.5mm}\\[1.0cm]
			{\huge \bfseries Exercises: Covariate Shift }\\[0.6cm]
			\rule{\linewidth}{0.5mm}\\[1.5cm]
			
			\begin{minipage}{0.4\textwidth}
				\begin{flushleft} \large	
					\emph{Author:}\\
					\textsc{Milan Niestijl, 4311728}
				\end{flushleft}
			\end{minipage}
			
			\vfill
			{\large \today}
		\end{center}
	\end{titlepage}
	
	\section*{1. Questions}
	
	\subsection*{1.1}
	---	Real life example ---
	
	\subsection*{1.2}
	The ratio might decay to zero very quickly, causing most of the samples in the target domain to receive (near) zero weight, which effectively reduces the number of training samples by a potentially drastic amount. This results in poor performance of the classifier.
	
	\subsection*{1.3}
	Let $\mathcal{S}$ and $\tau$ random variables corresponding to the source and target domain, respectively. Denote their probability density function by $p_{\mathcal{S}}(x)$ and $p_{\tau}(x)$. \\
	
	The sample average of the weights is constrained to be close to one since it approximates the integral of $p_{\tau}(x)$ over the sample space $\Omega$, which has to equal one by the definition of a probability space:
	\begin{align*}
		1 &= \int_{\Omega} p_{\tau}(x)dx\\
		&= \int_{\Omega} {p_{\tau}(x) \over p_{\mathcal{S}}(x)}p_{\mathcal{S}}(x)dx\\
		&= \int_{\Omega} w(x)p_{\mathcal{S}}(x)dx\\
		&= \mathbb{E}(w(\mathcal{S}))\\
		&\approx {1\over n} \sum_{i=1}^{N}  w(x_{i})
	\end{align*}
	
	\section*{2. Code Assignment}
	
	
	\bibliographystyle{authordate1}
	\begin{bibliography}{ref}
		
	\end{bibliography}

\end{document}