\documentclass [a4paper] {report}
\usepackage{amsmath,amssymb,amsthm, bbm, graphicx,listings,braket,subfig,titlesec,cleveref,lipsum,mcode,xcolor-patch, textcomp,float,booktabs,siunitx, listings}
\usepackage[authoryear]{natbib}
\usepackage[section]{placeins}
\usepackage[margin=2.2cm]{geometry}
\titleformat{\chapter}{\normalfont\huge}{\thechapter.}{20pt}{\huge \bf}

\DeclareMathOperator*{\argmin}{arg\,min}
\DeclareMathOperator*{\argmax}{arg\,max}
\newcommand{\norm}[1]{\left\lVert #1 \right\rVert}

\begin{document}
	
	\begin{titlepage}
		\begin{center}
			
			\textsc{\LARGE IN4320 Machine Learning}\\[1.25cm]
			
			\rule{\linewidth}{0.5mm}\\[1.0cm]
			{\huge \bfseries Exercise Computational Learning Theory: Boosting }\\[0.6cm]
			\rule{\linewidth}{0.5mm}\\[1.5cm]
			
			\begin{minipage}{0.4\textwidth}
				\begin{flushleft} \large	
					\emph{Author:}\\
					\textsc{Milan Niestijl, 4311728}
				\end{flushleft}
			\end{minipage}
			
			\vfill
			{\large \today}
		\end{center}
	\end{titlepage}
	
	\section*{a.}
	Let $x\geq0$. then $e^{-x} \geq (1-x)$.\\
	\subsection*{Proof:}
	Note that $e^{-x} \geq 0$ and $e^{0} = 1$, so that the result is trivial for $x\geq 1$ and $x=0$. We assume $x \in (0,1)$. Then we have, using the series representation of $e^{-x}$:
	\begin{equation*}
		\begin{split}
			e^{-x} - (1-x) &=\sum_{n=0}^{\infty}\frac{(-x)^{n}}{n!} - (1-x)\\
			&= \sum_{n=2}^{\infty}\frac{(-x)^{n}}{n!}\\
			&= \sum_{n=1}^{\infty}\frac{x^{2n}}{(2n)!} - \frac{(x)^{2n+1}}{(2n+1)!}\\
			&= \sum_{n=1}^{\infty}\frac{x^{2n}}{(2n)!}\left( 1 - \frac{x}{2n + 1} \right)\\
			&\geq 0\\
		\end{split}
	\end{equation*}
	Where in the last step it was used that each term of the sum is positive.

\end{document}